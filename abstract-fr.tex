La technologie sans fil est un catalyseur clé des promesses de l’industrie 4.0 (fabrication intelligente). En tant que telle, la technologie sans fil sera adoptée comme mode de communication principal au sein de l’usine en général et dans les unités de production en particulier. La communication des unités de production en usine a des exigences particulières en matière de latence, de fiabilité, d’échelle et de sécurité qui doivent d’abord être satisfaites par la technologie de communication sans fil utilisée. Le sans fil est considéré comme une forme de communication non idéale dans la mesure où par rapport aux communications câblées, il est considéré comme moins fiable (avec perte) et moins sécurisé. Ces dégradations possibles entraînent un retard et une perte de données dans un système d’automatisation industrielle où le déterminisme, la sécurité et la sûreté sont considérés comme primordiaux. Cette thèse étudie les exigences d’une communication sans fil dans les unités de production et l’applicabilité de la technologie sans fil existante dans ce domaine. Elle présente une modélisation SysML de l’architecture du système et des flux de données. Elle fournit une méthode d’utilisation des bases de données de type graphe pour l’organisation et l’analyse des données de performance collectées à partir d’un environnement de test. Enfin, la thèse décrit une approche utilisant l’apprentissage automatique pour l’évaluation des performances d’un système d’objets connectés dans le domaine de fabrication.  
%La technologie sans fil est un catalyseur clé des promesses de l'industrie 4.0 (fabrication intelligente). En tant que telle, la technologie sans fil sera adoptée comme mode de communication principal au sein de l'usine, en commençant par l'entreprise d'usine et finalement adoptée pour une utilisation au sein de la cellule de travail de l'usine. La communication des cellules de travail en usine a des exigences particulières en matière de latence, de fiabilité, d'échelle et de sécurité qui doivent d'abord être satisfaites par la technologie de communication sans fil utilisée. Le sans fil est considéré comme une forme de communication non idéale dans la mesure où, par rapport à ses homologues câblés, il est considéré comme moins fiable (avec perte) et moins sécurisé. Ces dégradations possibles entraînent un retard et une perte de données dans un système d'automatisation industrielle où le déterminisme, la sécurité et la sûreté sont considérés comme primordiaux. Cette thèse étudie les exigences sans fil de la cellule de travail de l'usine et l'applicabilité de la technologie sans fil existante, elle présente une approche de modélisation de la découverte de l'architecture et des flux de données à l'aide de SysML, elle fournit une méthode d'utilisation des bases de données graphiques pour l'organisation et l'analyse des données de performance collectés à partir d'un environnement de banc d'essai, et fournit enfin une approche de l'utilisation de l'apprentissage automatique dans l'évaluation des performances du système cyberphysique.