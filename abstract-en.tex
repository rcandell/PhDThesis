Wireless technology is a key enabler of the promises of Industry 4.0 (Smart Manufacturing). As such, wireless technology will be adopted as a principal mode of communication within the factory beginning with the factory enterprise and eventually being adopted for use within the factory workcell.  Factory workcell communication has particular requirements on latency, reliability, scale, and security that must first be met by the wireless communication technology used.  Wireless is considered a non-ideal form of communication in that when compared to its wired counterparts, it is considered less reliable (lossy) and less secure.  These possible impairments lead to delay and loss of data in industrial automation systems where determinism, security, and safety are considered paramount.  This thesis investigates the wireless requirements of the factory workcell and applicability of existing wireless technology, presents a modeling approach to discovery of architecture and data flows using SysML, provides a method for the use of graph databases to the organization and analysis of performance data collected from a testbed environment, and, finally, provides an approach to using machine learning in the evaluation of cyber-physical system performance.