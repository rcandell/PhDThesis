Cyber-physical systems (CPS) are defined as a holistic integration of computing, networking, and physical processes.  Implicit within these systems are feedback connections in which the computing and networking devices affect the physical systems directly.  Traditionally, industrial computation was performed using dedicated electronics with analog process variables and control signals being communicated over twisted-pair wires.  As automation capabilities evolved, serial communications architectures such as the common fieldbus were adopted.  With the advent of the Internet and Cloud Computing, Internet Protocol (IP) was adopted for more advanced data collection and control applications.  In such applications, reliability and latency requirements were not overly difficult to achieve given the existing technologies; however applications requiring tight feedback timing and reliability were not addressed as the protocols primarily targeted slower flow-based processes and building automation.  Discrete manufacturing requirements remained largely unaddressed.  To answer the call of discrete manufacturing requirements, routable industrial IP protocols such as Common Internet Protocol (CIP), SERCOS III, and Profinet by Siemens were developed to guarantee latency, reliability, and interoperability between systems.  However, given that these protocols are often IP-routable and share a common communication medium, these types of systems are limited in their ability to meet the performance demands of the physical systems or guarantee security.   As wired protocols, they can lack flexibility and mobility demanded by modern industrial applications.

Communication strategies that jointly address reliability, timing, scale, and power are needed to meet the needs of future industrial control systems.  Proponents of advanced manufacturing systems such as those defined by Industry 4.0 and the Industrial Internet Consortium state that existing protocols are not capable of meeting all the demands of industry.  The future factory will require untethered situationally-aware communications, heavy reliance on automation to include robotics, and an increased cognitive cybersecurity posture.  It is not certain if feedback control of robot motion is necessary; however future communication systems must address strict latency and data reliability requirements for discrete sensors, actuators, and robot end-effectors while maintaining security.  With the adoption of wireless communications in the factory, existing protocols such as IEEE 802.11 and IEEE 802.15.4 will not be capable of meeting reliability demands, round-trip latencies under 10ms, or scalability to dozens of devices within individual work-cells and hundreds or thousands of devices within an entire factory system.   
 
Wireless communication is inherently more prone to latency and delay than wired counterparts.  In addition, wireless communication implies the utilization of the electromagnetic spectrum which is a publicly accessible medium with constrained capacity and more prone to cyber-attack.  While transmitted data can by digitally protected through authentication and encryption, wireless devices are prone to interference and jamming by both rogue and friendly emitters exacerbating the reliability and latency concern impacting factory performance without compromising data security.  Wireless communication in factories is often constrained by battery life and most certainly constrained by the availability of the electromagnetic spectrum.   

As the reliance on wireless devices within the factory continues, three steps toward developing a more robust wireless factory communications network must be developed. The automation system must become situationally aware and adaptive to knowledge of the trends in electromagnetic spectrum occupancy and acute events; Intelligence of the automation system must move closer to the physical system.  This means moving the intelligence for control to the actuator; Performance test methods must be developed and incorporated into the industrial fringe devices.  The test methods must be dependable and at the same time easy to use by factory personal not trained in the technicalities of wireless communication; Existing Wireless communications protocols must be analyzed and adapted, and new protocols must be developed to balance reliability, latency, and scalability; Security of the network must be maintained and must include availability as a paramount characteristic.

The proposed thesis will include development of standard test methods for measuring the performance of industrial wireless networks deployed within smart manufacturing work-cells.  The thesis work will focus on robot end-effector actuation and discrete sensing and actuation within a work-work-cell.  A literature survey will be an ongoing activity to ascertain the state-of-the-art of control of manufacturing processes over non-ideal communication networks such as wireless networks.  The survey will include situational awareness strategies that incorporate spectrum sensing and cybersecurity into the controller algorithm.  The thesis will design protocols that facilitate the integration of situational sensing systems with the control application.  Technological strategies will be developed to move intelligence closer to actuation devices thereby eliminating centralized control.  Centralized control has many advantages, but presents a single point of failure with respect to interference, jamming, and cyber-attack.  The primary motivation of the research will be to create a practical framework for incorporating a complete situational awareness picture into the control architecture thereby improving security, safety, and reliability.   Findings and results of the thesis work will be included within the thesis and published as journal and conference references.  Protocols, test methods, artifacts, and algorithms will be proposed for inclusion in applicable CPS standards. 
